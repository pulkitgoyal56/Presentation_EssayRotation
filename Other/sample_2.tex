%----------------------------------------------------------------------------------------
%	PACKAGES AND THEMES
%----------------------------------------------------------------------------------------
\documentclass[aspectratio=169,xcolor=dvipsnames]{beamer}
\usetheme{Madrid}

\usepackage{hyperref}
\usepackage{graphicx} % Allows including images
\usepackage{booktabs} % Allows the use of \toprule, \midrule and \bottomrule in tables

% \beamerdefaultoverlayspecification{<+->}
\setbeamerfont{footnote}{size=\tiny}

\usepackage[style=authoryear,backend=biber]{biblatex}
% \setbeamertemplate{bibliography item}{}
\renewcommand*{\bibfont}{\scriptsize{}}
\setlength{\bibhang}{0pt}
% \makeatletter
%   \mode<presentation>{
%       \newlength{\beamer@bibiconwidth}
%       \settowidth\beamer@bibiconwidth{\usebeamertemplate*{bibliography item}}
%       \setlength{\labelwidth}{-\beamer@bibiconwidth}
%       \addtolength{\labelwidth}{2\labelsep}
%       \addtolength{\bibhang}{\labelsep}
%   }
% \makeatother
\addbibresource{ref.bib} % \bibliography{ref.bib}

%----------------------------------------------------------------------------------------
%	TITLE PAGE
%----------------------------------------------------------------------------------------
\title[]{Can Self-Exploring RL Agents Have Compulsive Behavior?}
\subtitle{\footnotesize Essay Rotation}

\author[]{\Large Pulkit Goyal}
\institute[]
{
    M.Sc. Student, Neural Information Processing \\
    University of Tübingen
    \vskip 12pt

    \textit{Supervised by} \\\vspace{2pt}
    {\footnotesize Cansu Sancaktar}\\
    Autonomous Learning Group, MPI-IS
}
\date{\today}

%----------------------------------------------------------------------------------------
%	PRESENTATION SLIDES
%----------------------------------------------------------------------------------------
\begin{document}

%************************************************
\begin{frame}
    % Print the title page as the first slide
    \titlepage
\end{frame}
%************************************************

%************************************************
\begin{frame}{Overview}
    % Throughout your presentation, if you choose to use \section{} and \subsection{} commands, these will automatically be printed on this slide as an overview of your presentation
    \setcounter{tocdepth}{2}
    \tableofcontents % [pausesections]
\end{frame}
%************************************************

%************************************************
\section{Obsessive-Compulsive Disorder}
%------------------------------------------------
\subsection{What is OCD?}

%------------------------------------------------
\subsection{Low-Level Characteristics of OCD}

%************************************************

%************************************************
\section{A Computational Model of OCD} %  | Excessive Uncertainty Regarding State Transitions
%------------------------------------------------


%************************************************

%************************************************
\section{Bayesian-Brain Framework}
%------------------------------------------------
\subsection{Free-Energy Principle}
\subsubsection{Predictive Coding and Active Inference}

%------------------------------------------------
\subsection{Ties to Self-Supervised Reinforcement Learning}

%------------------------------------------------
\subsubsection{VIME}

%************************************************

%************************************************
\section{Computational Models of Related Mental Disorders}
%------------------------------------------------
\subsection{Schizophrenia}

%------------------------------------------------
\subsection{Autism Spectrum Disorder (ASD)}

%************************************************

%************************************************
\section{Modeling Obsessive-Compulsive Behavior in Self-Exploring RL Agents} % Inspiration from mentioned studies
%------------------------------------------------
% \subsection{Proposals}

\begin{frame}{Frame Title}
    
\end{frame}

\note[itemize]{
\item point 1
\item point 2
}

%************************************************



%------------------------------------------------

% \begin{frame}{Frame Title}
%     \begin{enumerate}[<+->]
%         \item ()
%         \item ()
%     \end{enumerate}
% \end{frame}

%------------------------------------------------

% \begin{columns}[c] % [c] or [t]
%     \column{.45\textwidth} % Left column and width
%     \column{.5\textwidth} % Right column and width
% \end{columns}

%------------------------------------------------

% \begin{table}
%     \begin{tabular}{l l l}
%         \toprule
%         \textbf{} & \textbf{} & \textbf{} \\
%         \midrule
%         ()         & ()           & ()               \\
%         \bottomrule
%     \end{tabular}
%     \caption{}
% \end{table}

%------------------------------------------------

% \begin{frame}{Theorem}
%     \begin{theorem}[]
%         $$
%     \end{theorem}
% \end{frame}

%------------------------------------------------

% \begin{figure}
%     \includegraphics[width=0.8\linewidth]{}
% \end{figure}

%------------------------------------------------

% \begin{frame}[fragile] % Need to use the fragile option when verbatim is used in the slide
%     \frametitle{Citation}
%     An example of the \verb|\cite| command to cite within the presentation:\\~
%     This statement requires citation \footcite{pl, yl}.
% \end{frame}

%------------------------------------------------

\begin{frame}{References} % [allowframebreaks]
    % Beamer does not support BibTeX so references must be inserted manually as below
    % \footnotesize{
    %     \begin{thebibliography}{99}
    %         \bibitem[Smith, 2012]{p1} John Smith (2012)
    %         \newblock Title of the publication
    %         \newblock \emph{Journal Name} 12(3), 45 -- 678.
    %     \end{thebibliography}
    % }
    % \bibliographystyle{apalike}
    \printbibliography
\end{frame}

%------------------------------------------------

\begin{frame}
    \Huge{\centerline{Thank You!}}
\end{frame}

%----------------------------------------------------------------------------------------

\end{document}